\documentclass[
	% -- opções da classe memoir --
	12pt,			% tamanho da fonte
	openany,			% capítulos começam em pág ímpar - openright 
					% independente de onde começa a página - openany
	oneside,			% para impressão em recto e verso - twoside.
					% para impressão apenas frente - oneside
	a4paper,			% tamanho do papel. 
	% -- opções da classe abntex2 --
	%chapter=TITLE,		% títulos de capítulos convertidos em letras maiúsculas
	%section=TITLE,		% títulos de seções convertidos em letras maiúsculas
	%subsection=TITLE,	% títulos de subseções convertidos em letras maiúsculas
	%subsubsection=TITLE,% títulos de subsubseções convertidos em letras maiúsculas
	% -- opções do pacote babel --
	english,			% idioma adicional para hifenização
	french,				% idioma adicional para hifenização
	spanish,			% idioma adicional para hifenização
	brazil,				% o último idioma é o principal do documento
	fleqn			% alinhamento das equações à esquerda
	]{abntex2}

% ---
% Pacotes básicos 
% ---
\usepackage{lmodern}			% Usa a fonte Latin Modern			
\usepackage[T1]{fontenc}		% Selecao de codigos de fonte.
\usepackage[utf8]{inputenc}	% Codificacao do documento (acentuação)
\usepackage{indentfirst}		% Indenta o primeiro parágrafo de cada seção.
\usepackage{color}			% Controle das cores
\usepackage{graphicx}			% Inclusão de gráficos
\usepackage{microtype} 		% para melhorias de justificação
% ---
		
% ---
% Pacotes adicionais, usados apenas no âmbito do Modelo Canônico do abnteX2
% ---
\usepackage{lipsum}				% para geração de dummy text
% ---

% ---
% Pacotes de citações
% ---
% \usepackage[brazilian,hyperpageref]{backref}	 % Paginas com as citações na bibl
\usepackage[alf]{abntex2cite}	% Citações padrão ABNT

% --- 
% CONFIGURAÇÕES DE PACOTES
% --- 

% ---
% Configurações do pacote backref
% Usado sem a opção hyperpageref de backref
% \renewcommand{\backrefpagesname}{Citado na(s) página(s):~}
% Texto padrão antes do número das páginas
% \renewcommand{\backref}{}
% Define os textos da citação
% \renewcommand*{\backrefalt}[4]{
% 	\ifcase #1 %
% 		Nenhuma citação no texto.%
% 	\or
% 		Citado na página #2.%
% 	\else
% 		Citado #1 vezes nas páginas #2.%
% 	\fi}%
% ---

% ---
% Informações de dados para CAPA e FOLHA DE ROSTO
% ---
\titulo{Título do seu trabalho}
\autor{Seu nome}
\local{Natal/RN}
\data{ mês, ano}
\orientador{Prof. Dr. Fulano de tal } %deixar espaço
\coorientador{Prof. Dr. Cicrano de tal}
\instituicao{%
  UNIVERSIDADE FEDERAL DO RIO GRANDE DO NORTE
  \par
  CENTRO DE TECNOLOGIA
  \par
  PROGRAMA DE PÓS-GRADUAÇÃO EM ENGENHARIA QUÍMICA}
\tipotrabalho{Seu tipo de trabalho } %deixar espaço
%Dissertação (mestrado); tese (doutorado); proposta de dissertação/tese

% O preambulo deve conter o tipo do trabalho, o objetivo, 
% o nome da instituição e a área de concentração 
\preambulo{\imprimirtipotrabalho apresentada ao Programa de Pós-graduação em Engenharia Química (PPGEQ) da Universidade Federal do Rio Grande do Norte (UFRN) como parte dos requisitos necessários à obtenção do grau de Mestre/Doutor, sob a
orientação do \imprimirorientador e coorientação do \imprimircoorientador.}
% ---


% ---
% Configurações de aparência do PDF final

% alterando o aspecto da cor azul
\definecolor{blue}{RGB}{41,5,195}

% informações do PDF
\makeatletter
\hypersetup{
     	%pagebackref=true,
		pdftitle={\@title}, 
		pdfauthor={\@author},
    	pdfsubject={\imprimirpreambulo},
	    pdfcreator={LaTeX with abnTeX2},
		pdfkeywords={abnt}{latex}{abntex}{abntex2}{trabalho acadêmico}, 
		colorlinks=true,       		% false: boxed links; true: colored links
    	linkcolor=blue,          	% color of internal links
    	citecolor=blue,        		% color of links to bibliography
    	filecolor=magenta,      		% color of file links
		urlcolor=blue,
		bookmarksdepth=4
}
\makeatother
% --- 

% ---
% Posiciona figuras e tabelas no topo da página quando adicionadas sozinhas
% em um página em branco. Ver https://github.com/abntex/abntex2/issues/170
\makeatletter
\setlength{\@fptop}{5pt} % Set distance from top of page to first float
\makeatother
% ---

% ---
% Possibilita criação de Quadros e Lista de quadros.
% Ver https://github.com/abntex/abntex2/issues/176
%
\newcommand{\quadroname}{Quadro}
\newcommand{\listofquadrosname}{Lista de quadros}

\newfloat[chapter]{quadro}{loq}{\quadroname}
\newlistof{listofquadros}{loq}{\listofquadrosname}
\newlistentry{quadro}{loq}{0}

% configurações para atender às regras da ABNT
\setfloatadjustment{quadro}{\centering}
\counterwithout{quadro}{chapter}
\renewcommand{\cftquadroname}{\quadroname\space} 
\renewcommand*{\cftquadroaftersnum}{\hfill--\hfill}

\setfloatlocations{quadro}{hbtp} % Ver https://github.com/abntex/abntex2/issues/176
% ---

% --- 
% Espaçamentos entre linhas e parágrafos 
% --- 

% O tamanho do parágrafo é dado por:
\setlength{\parindent}{1.3cm}

% Controle do espaçamento entre um parágrafo e outro:
\setlength{\parskip}{0.2cm}  % tente também \onelineskip

% ---
% compila o indice
% ---
\makeindex
% ---

%%criar um novo estilo de cabeçalhos e rodapés
\makepagestyle{meuestilo}
%%cabeçalhos
\makeevenhead{meuestilo} %%pagina par
{}
{}
{}
\makeoddhead{meuestilo} %%pagina ímpar ou com oneside
{\rightmark}
{}
{}

\makeheadrule{meuestilo}{\textwidth}{\normalrulethickness}

%% rodapé
\makeevenfoot{meuestilo}
{} %%pagina par
{}
{} 
\makeoddfoot{meuestilo} %%pagina ímpar ou com oneside
{\imprimirautor}
{}
{\thepage}

\makefootrule{meuestilo}{\textwidth}{\normalrulethickness}{0pt}
% Novo estilo usado apenas no cabeçalho da página em que se inicia o capítulo

\makepagestyle{meuestilo_capitulo}
%%cabeçalhos
\makeevenhead{meuestilo_capitulo} %%pagina par
{}
{}
{}
\makeoddhead{meuestilo_capitulo} %%pagina ímpar ou com oneside
{\rightmark}
{}
{}

\makeheadrule{meuestilo_capitulo}{\textwidth}{\normalrulethickness}

%% rodapé
\makeevenfoot{meuestilo_capitulo}
{} %%pagina par
{}
{} 
\makeoddfoot{meuestilo_capitulo} %%pagina ímpar ou com oneside
{\imprimirautor}
{}
{\thepage}

\makefootrule{meuestilo_capitulo}{\textwidth}{\normalrulethickness}{0pt}

%%% -----
%%% Formato de cabeçalho/rodapé romano nos elementos pré-textuais
%%% -----

%% Novo estilo
\makepagestyle{estilo_pretextual} %%% escolha um nome
\makeevenfoot{estilo_pretextual}{}{}{\ABNTEXfontereduzida \textsc \thepage}
\makeoddfoot{estilo_pretextual}{}{}{\ABNTEXfontereduzida \textsc \thepage}

%% Novo estilo \part
\makepagestyle{estilo_part} %%% escolha um nome
\makeevenfoot{estilo_part}{}{}{}
\makeoddfoot{estilo_part}{}{}{}

%% Customiza comando \pretextual
\renewcommand{\pretextual}{
	\pagenumbering{roman} %%% ou \pagenumbering{Roman}
	\aliaspagestyle{chapter}{estilo_pretextual}% customizing chapter pagestyle
	\pagestyle{estilo_pretextual}
	\aliaspagestyle{cleared}{empty}
	\aliaspagestyle{part}{estilo_part}
}

% ---
% Ajusta a marca \textual para que a numeração volte a ser arábica
% nos elementos textuais
\let\oldtextual\textual        % copia o comando \textual anterior para \oldtextual
\renewcommand{\textual}{%
	\oldtextual%
	\pagenumbering{arabic} % volta à numeração arábica
}

% ----
% Início do documento
% ----
\begin{document}

% Seleciona o idioma do documento (conforme pacotes do babel)
%\selectlanguage{english}
\selectlanguage{brazil}

% ---
% ---
% Capa
% ---
% ---

\thispagestyle{empty}

\begin{center}
	\begin{figure}[!htb]
		\centering
		\includegraphics[width=3cm]{Figuras/ufrn.jpg}
		\hspace*{\fill}
		\includegraphics[width=4cm]{Figuras/ppgeq.jpg}
		%\caption{Legenda}
		%\label{Rotulo}
	\end{figure}
	\imprimirinstituicao \\
	\vspace*{\fill} 
	\imprimirtipotrabalho\\
	\vspace*{\fill} 
	\begin{center}
		\ABNTEXchapterfont\bfseries\large\imprimirtitulo
	\end{center}
	\vspace*{\fill} 
	\imprimirautor \\
	\vspace*{\fill} 
	\imprimirorientadorRotulo: \imprimirorientador \\
	\imprimircoorientadorRotulo: \imprimircoorientador \\
	\vspace*{\fill}
	\begin{center}
		\vspace*{0.5cm}
		{\normalsize\imprimirlocal}
		\par
		{\normalsize\imprimirdata}
		\vspace*{1cm}
	\end{center}
\end{center}
% ---
% ---
% Folha de rosto
% ---
% ---
\newpage
\begin{center}
	{\ABNTEXchapterfont\large\imprimirautor}
	\vspace*{\fill}\vspace*{\fill}
	\begin{center}
		\ABNTEXchapterfont\bfseries\large\imprimirtitulo
	\end{center}
	\vspace*{\fill}
	
	\hspace{.45\textwidth}
	\begin{minipage}{.5\textwidth}
		\imprimirpreambulo
	\end{minipage}%
	\vspace*{\fill}
	
	\begin{center}
		\vspace*{0.5cm}
		{\imprimirlocal}
		\par
		{\imprimirdata}
		\vspace*{1cm}
	\end{center}
\end{center}

% ---

% ---
% Inserir a ficha bibliografica
% ---

% Isto é um exemplo de Ficha Catalográfica, ou ``Dados internacionais de
% catalogação-na-publicação''. Você pode utilizar este modelo como referência. 
% Porém, provavelmente a biblioteca da sua universidade lhe fornecerá um PDF
% com a ficha catalográfica definitiva após a defesa do trabalho. Quando estiver
% com o documento, salve-o como PDF no diretório do seu projeto e substitua todo
% o conteúdo de implementação deste arquivo pelo comando abaixo:
%
% \begin{fichacatalografica}
%     \includepdf{fig_ficha_catalografica.pdf}
% \end{fichacatalografica}
\newpage
\begin{fichacatalografica}
	\sffamily
	\vspace*{\fill}					% Posição vertical
	\begin{center}					% Minipage Centralizado
	\fbox{\begin{minipage}[c][8cm]{13.5cm}		% Largura
	\small
	\imprimirautor
	%Sobrenome, Nome do autor
	
	\hspace{0.5cm} \imprimirtitulo  / \imprimirautor. --
	\imprimirlocal, \imprimirdata-
	
	\hspace{0.5cm} \thelastpage p. : il.\\
	
	\hspace{0.5cm} \imprimirorientadorRotulo~\imprimirorientador\\
	
	\hspace{0.5cm}
	\parbox[t]{\textwidth}{\imprimirtipotrabalho~--~Universidade Federal do Rio Grande do \\Norte. Centro de Tecnologia. Programa de Pós-graduação em Engenharia \\Química,
	\imprimirdata.}\\
	
	\hspace{0.5cm}
		1. Biodiesel.
		2. Reator Spray.
		2. Transesterificação.			
	\end{minipage}}
	\end{center}
\end{fichacatalografica}
% ---

% ---
% Inserir errata
% ---
\begin{comment}

\begin{errata}
Elemento opcional da \citeonline[4.2.1.2]{NBR14724:2011}. Exemplo:

\vspace{\onelineskip}

FERRIGNO, C. R. A. \textbf{Tratamento de neoplasias ósseas apendiculares com
reimplantação de enxerto ósseo autólogo autoclavado associado ao plasma
rico em plaquetas}: estudo crítico na cirurgia de preservação de membro em
cães. 2011. 128 f. Tese (Livre-Docência) - Faculdade de Medicina Veterinária e
Zootecnia, Universidade de São Paulo, São Paulo, 2011.

\begin{table}[htb]
\center
\footnotesize
\begin{tabular}{|p{1.4cm}|p{1cm}|p{3cm}|p{3cm}|}
  \hline
   \textbf{Folha} & \textbf{Linha}  & \textbf{Onde se lê}  & \textbf{Leia-se}  \\
    \hline
    1 & 10 & auto-conclavo & autoconclavo\\
   \hline
\end{tabular}
\end{table}

\end{errata}
\end{comment}
% ---

% ---
% Inserir folha de aprovação
% ---

% Isto é um exemplo de Folha de aprovação, elemento obrigatório da NBR
% 14724/2011 (seção 4.2.1.3). Você pode utilizar este modelo até a aprovação
% do trabalho. Após isso, substitua todo o conteúdo deste arquivo por uma
% imagem da página assinada pela banca com o comando abaixo:
%
% \begin{folhadeaprovacao}
% \includepdf{folhadeaprovacao_final.pdf}
% \end{folhadeaprovacao}
%
\begin{folhadeaprovacao}

  \begin{center}
    {\ABNTEXchapterfont\large\imprimirautor}

    \vspace*{\fill}\vspace*{\fill}
    \begin{center}
      \ABNTEXchapterfont\bfseries\large\imprimirtitulo
    \end{center}
    \vspace*{\fill}
    
    \hspace{.45\textwidth}
    \begin{minipage}{.5\textwidth}
        \imprimirpreambulo
    \end{minipage}%
    \vspace*{\fill}
  \end{center}
        
   \imprimirtipotrabalho apresentada em \imprimirlocal, [DIA] de [MÊS] de [ANO].

   \assinatura{\textbf{\imprimirorientador} \\ Orientador} 
   \assinatura{\textbf{\imprimircoorientador} \\ Coorientador}
   %\assinatura{\textbf{Dr. } \\ Membro externo}
   %\assinatura{\textbf{Professor} \\ Convidado 3}
   %\assinatura{\textbf{Professor} \\ Convidado 4}
      
   \begin{center}
    \vspace*{0.5cm}
    {\imprimirlocal}
    \par
    {\imprimirdata}
    \vspace*{1cm}
  \end{center}
  
\end{folhadeaprovacao}
% ---
% ---
% Dedicatória
% ---
% ---
% \begin{dedicatoria}
% \vspace*{\fill}
% \centering
% \noindent
% \textit{ Este trabalho é dedicado às crianças adultas que,\\
% quando pequenas, sonharam em se tornar cientistas.} \vspace*{\fill}
% \end{dedicatoria}
% ---
% ---
% Agradecimentos
% ---
% ---
\newpage

\begin{agradecimentos}
Os agradecimentos principais são direcionados à Gerald Weber, Miguel Frasson,
Leslie H. Watter, Bruno Parente Lima, Flávio de Vasconcellos Corrêa, Otavio Real
Salvador, Renato Machnievscz\footnote{Os nomes dos integrantes do primeiro
projeto abn\TeX\ foram extraídos de
\url{http://codigolivre.org.br/projects/abntex/}} e todos aqueles que
contribuíram para que a produção de trabalhos acadêmicos conforme
as normas ABNT com \LaTeX\ fosse possível.

Agradecimentos especiais são direcionados ao Centro de Pesquisa em Arquitetura
da Informação\footnote{\url{http://www.cpai.unb.br/}} da Universidade de
Brasília (CPAI), ao grupo de usuários
\emph{latex-br}\footnote{\url{http://groups.google.com/group/latex-br}} e aos
novos voluntários do grupo
\emph{\abnTeX}\footnote{\url{http://groups.google.com/group/abntex2} e
\url{http://www.abntex.net.br/}}~que contribuíram e que ainda
contribuirão para a evolução do \abnTeX
\end{agradecimentos}



% ---
% ---
% Epígrafe
% ---
% ---
\newpage
\vspace*{\fill}
\begin{flushright}
	\textit{``
		Um perito é alguém que cometeu \\ todos os erros possíveis\\
		numa determinada área específica.`` \\
		(Niels Bohr)}
\end{flushright}
\newpage
% ---

% ---
% RESUMOS
% ---

% resumo em português
\setlength{\absparsep}{18pt} % ajusta o espaçamento dos parágrafos do resumo
\begin{resumo}
 Segundo a \citeonline[3.1-3.2]{NBR6028:2003}, o resumo deve ressaltar o
 objetivo, o método, os resultados e as conclusões do documento. A ordem e a extensão
 destes itens dependem do tipo de resumo (informativo ou indicativo) e do
 tratamento que cada item recebe no documento original. O resumo deve ser
 precedido da referência do documento, com exceção do resumo inserido no
 próprio documento. (\ldots) As palavras-chave devem figurar logo abaixo do
 resumo, antecedidas da expressão Palavras-chave:, separadas entre si por
 ponto e finalizadas também por ponto.

 \textbf{Palavras-chave}: latex. abntex. editoração de texto.
\end{resumo}

% resumo em inglês
\begin{resumo}[Abstract]
 \begin{otherlanguage*}{english}
   This is the english abstract.

   \vspace{\onelineskip}
 
   \noindent 
   \textbf{Keywords}: latex. abntex. text editoration.
 \end{otherlanguage*}
\end{resumo}

% resumo em francês 
%\begin{resumo}[Résumé]
 %\begin{otherlanguage*}{french}
 %   Il s'agit d'un résumé en français.
 
 %  \textbf{Mots-clés}: latex. abntex. publication de textes.
% \end{otherlanguage*}
%\end{resumo}

% resumo em espanhol
%\begin{resumo}[Resumen]
 %\begin{otherlanguage*}{spanish}
 %  Este es el resumen en español.
  
 %  \textbf{Palabras clave}: latex. abntex. publicación de textos.
% \end{otherlanguage*}
% \end{resumo}
% ---

% ---
% inserir lista de ilustrações
% ---
\pdfbookmark[0]{\listfigurename}{lof}
\listoffigures*
\cleardoublepage
% ---

% ---
% inserir lista de quadros
% ---
%\pdfbookmark[0]{\listofquadrosname}{loq}
%\listofquadros*
%\cleardoublepage
% ---

% ---
% inserir lista de tabelas
% ---
\pdfbookmark[0]{\listtablename}{lot}
\listoftables*
\cleardoublepage
% ---

% ---
% inserir lista de abreviaturas e siglas
% ---
\begin{siglas}
  \item[ABNT] Associação Brasileira de Normas Técnicas
  \item[abnTeX] ABsurdas Normas para TeX
\end{siglas}
% ---

% ---
% inserir lista de símbolos
% ---
\begin{simbolos}
  \item[$ \Gamma $] Letra grega Gama
  \item[$ \Lambda $] Lambda
  \item[$ \zeta $] Letra grega minúscula zeta
  \item[$ \in $] Pertence
\end{simbolos}
% ---

% ---
% inserir o sumario
% ---
\pdfbookmark[0]{\contentsname}{toc}
\tableofcontents*
\cleardoublepage
% ---



% ----------------------------------------------------------
% ELEMENTOS TEXTUAIS
% ----------------------------------------------------------
\textual
\pagestyle{meuestilo}
\aliaspagestyle{chapter}{meuestilo_capitulo}% customizing chapter pagestyle
% ----------------------------------------------------------
% PARTE
% ----------------------------------------------------------
\stepcounter{chapter}
\part*{Capítulo \thechapter \\ Introdução}
\addtocounter{chapter}{-1}
% ----------------------------------------------------------

% ----------------------------------------------------------
% Introdução (exemplo de capítulo sem numeração, mas presente no Sumário)
% ----------------------------------------------------------
\chapter{Introdução}
% ----------------------------------------------------------

\chapterprecis{Isto é uma sinopse de capítulo. Um curto parágrafo introdutório é essencial no início de cada capítulo.}\index{sinopse de capítulo}

Apresentar o assunto a ser estudado, abordando os aspectos gerais e buscando
introduzir o leitor na temática delineada. Também, fazer uma descrição sucinta dos
objetivos do trabalho, podendo ser divididos em objetivo geral e objetivos específicos.
Ressaltar a importância do trabalho dentro de um contexto científico e tecnológico,
relatando as possíveis contribuições dos resultados alcançados.




% ----------------------------------------------------------
% PARTE
% ----------------------------------------------------------
\stepcounter{chapter}
\part*{Capítulo \thechapter \\ Revisão Bibliográfica}
\addtocounter{chapter}{-1}
% ----------------------------------------------------------

% ---
% Capitulo de revisão de literatura
% ---
\chapter{ Revisão Bibliográfica}
% ---

\chapterprecis{Isto é uma sinopse de capítulo. Um curto parágrafo introdutório é essencial no início de cada capítulo.}\index{sinopse de capítulo}

Abordar os aspectos teóricos diretamente relacionados com o trabalho desenvolvido,
detalhando os assuntos principais do estudo em questão e baseando-se nas diferentes
abordagens pesquisadas na literatura (livros, teses, dissertações, artigos, trabalhos de
congresso, etc.).

% ----------------------------------------------------------
% PARTE
% ----------------------------------------------------------
\stepcounter{chapter}
\part*{Capítulo \thechapter \\ Metodologia }
\addtocounter{chapter}{-1}

% ----------------------------------------------------------

% ---
% Capitulo com exemplos de comandos inseridos de arquivo externo 
% ---
\include{abntex2-modelo-include-comandos}
% ---
\chapter{Metodologia}\label{cap_trabalho_academico}

\chapterprecis{Isto é uma sinopse de capítulo. Um curto parágrafo introdutório é essencial no início de cada capítulo.}\index{sinopse de capítulo}

Apresentar os materiais e equipamentos utilizados na pesquisa experimental,
detalhando os métodos e procedimentos empregados durante as atividades, mostrando,
quando couber, diagramas das etapas executadas. No caso de trabalhos teórico-
computacionais, descrever o desenvolvimento dos modelos matemáticos, detalhando a
metodologia matemática ou numérica utilizada na resolução do modelo, os valores
considerados dos parâmetros do modelo, os equipamentos e softwares usados no estudo.

% ----------------------------------------------------------
% PARTE
% ----------------------------------------------------------
\stepcounter{chapter}
\part*{Capítulo \thechapter \\ Resultados e Discussão }
\addtocounter{chapter}{-1}
% ----------------------------------------------------------

% ---
% primeiro capitulo de Resultados
% ---
\chapter{Resultados e Discussão}
% ---

\chapterprecis{Isto é uma sinopse de capítulo. Um curto parágrafo introdutório é essencial no início de cada capítulo.}\index{sinopse de capítulo}

Apresentar os resultados, analisando e discutindo os diversos aspectos de interesse.

% ----------------------------------------------------------
% Finaliza a parte no bookmark do PDF
% para que se inicie o bookmark na raiz
% e adiciona espaço de parte no Sumário
% ----------------------------------------------------------
\phantompart
% ----------------------------------------------------------
% PARTE
% ----------------------------------------------------------

\stepcounter{chapter}
\part*{Capítulo \thechapter \\ Conclusão}
\addtocounter{chapter}{-1}
% ----------------------------------------------------------
% ---
% Conclusão
% ---
\chapter{Conclusão}
% ---

\chapterprecis{Isto é uma sinopse de capítulo. Um curto parágrafo introdutório é essencial no início de cada capítulo.}\index{sinopse de capítulo}

Relacionar as conclusões obtidas de acordo com os resultados obtidos no trabalho,
podendo incluir sugestões para trabalhos futuros.

% ----------------------------------------------------------
% ELEMENTOS PÓS-TEXTUAIS
% ----------------------------------------------------------
\postextual
% ----------------------------------------------------------

% ----------------------------------------------------------
% Referências bibliográficas
% ----------------------------------------------------------
\bibliography{abntex2-modelo-references}
% ----------------------------------------------------------
% Glossário
% ----------------------------------------------------------
%
% Consulte o manual da classe abntex2 para orientações sobre o glossário.
%
%\glossary

% ----------------------------------------------------------
% Apêndices
% ----------------------------------------------------------

% ---
% Inicia os apêndices
% ---
\begin{apendicesenv}

% Imprime uma página indicando o início dos apêndices
\partapendices

% ----------------------------------------------------------
\chapter{Quisque libero justo}
% ----------------------------------------------------------

\lipsum[50]

\end{apendicesenv}
% ---


% ----------------------------------------------------------
% Anexos
% ----------------------------------------------------------

% ---
% Inicia os anexos
% ---
\begin{anexosenv}

% Imprime uma página indicando o início dos anexos
\partanexos

% ---
\chapter{Instruções}
% ---
Complementar, na forma de uma sessão de anexos, assuntos que foram abordados no
corpo do texto, mas que apresentam aspectos particulares que não justificam incorporar seu
conteúdo como parte do trabalho de fato desenvolvido, servindo apenas de fundamentação,
comprovação ou ilustração.

\end{anexosenv}

\end{document}
